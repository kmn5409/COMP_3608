\documentclass{article}
\usepackage{parselines}
\pagestyle{plain}
\usepackage{amsmath}
\usepackage{amssymb}
\begin{document}

\begin{parse lines}[\noindent]{#1\\}

We would first imeplment one hot encoding to the features that are categorial


For example the feature:

"cut" -  has labels fair, good, very good, permium, ideal

The label "fair" using one hot encoding becomes a feature vector such as:

[1, 0, 0, 0, 0, 0]

The label "good" using one hot encoding becomes a feature vector such as:

[0, 1, 0, 0, 0, 0]


We would then feed each of the features into

 $$\frac{1}{n} \sum_{i=1}^{n} (\hat{y_i} - (w^Tx_i + c))^2$$

where $x_i$ composes of all the features for a particular diamond

For instance some of the features that would be put into the equation would be:

carat weight = 0.4

cut - [1, 0, 0, 0, 0]

z = 0.5

$\therefore$ $x_1$ = [0.4, 1, 0, 0, 0, 0, 0.5 etc...]
\end{parse lines}
\end{document}
